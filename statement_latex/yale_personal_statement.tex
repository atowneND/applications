\documentstyle[12pt]{article}
\setlength{\oddsidemargin}{0in}
\setlength{\evensidemargin}{0in}
\setlength{\textwidth}{6.5in}
\setlength{\topmargin}{-.3in}
\setlength{\textheight}{9in}
\pagestyle{empty}

\begin{document}

\begin{center}
{\Large Personal Statement} \\[.3in]
{\large Ashley Towne}
\end{center}

\vspace*{.5in}
{I am fascinated by the endeavor to physically understand everything that
    exists, but I also crave the practical applications of engineering and its
    associated problem solving process. These two realms of scientific thought
    combine at the intersection of electrical engineering and physics. To
    continue my pursuit of this intersection, I want to pursue a PhD in an area
    related to quantum computing and become a leader in the field. 

    I discovered my interest in quantum mechanics through two semesters of
    courses devoted to its study in addition to higher-level physics courses
    that include quantum mechanical components. My passion for research started
    with an affection for figuring out how tiny particles behave; although
    their “misbehavior” is certainly far more intriguing as it is the prelude
    to new understanding of physical phenomena. During Summer 2013, I worked on
    electrical engineering research. The process of approaching a problem,
    formulating questions in order to solve it, and then determining the best
    way to answer those questions was an intellectual challenge that I
    relished. I had discovered the joy of being on the frontier of knowledge.

    Last summer, I participated in a research experience for undergrads (REU)
    funded by the National Science Foundation examining rotationally inelastic
    collisions between He and NaK or Ar and NaK. The analysis involved a
    combination of shell scripts, C, gnuplot, and Fortran code to calculate and
    plot metrics that describe what happened during a collision, namely the
    angular distribution of probabilities of angular momentum transferred.  The
    goal was to develop an understanding of transfers of angular momentum that
    is simpler and easier to compute than the raw quantum mechanics describe.
    This semiclassical description based on the vector model of angular
    momentum resulted in a paper which has been submitted for publication.
    This REU confirmed for me that research is exactly what I want to do.  I
    want to become an expert on something interesting and make a significant
    and meaningful contribution.

    Despite my inclinations toward the systematic exploration and deep
    understanding of natural phenomena that research provides, I am drawn to
    technologies and problems that have meaningful, real world applications. My
    engineering background and internships in industry have made it clear to me
    that I find great personal fulfillment in working with technology that
    impacts the way the world works. During Summer 2014, I worked for Adaptive
    Methods doing signal processing work for sonar applications. Perhaps the
    most valuable lesson from this internship was the value of higher education
    in industry. It opens doors to leadership positions for deciding which
    technical challenges are most important and develops a higher level of
    critical thinking and problem solving. My goal is to make meaningful
    contributions to the field of quantum computing. Pursuing a PhD affords me
    the opportunity to conduct research and collaborate with other great
    scientific minds in order to make significant strides in this field.

    My coursework has also prepared me well, giving me a solid foundation of
    technical knowledge to succeed in graduate school and beyond. I have
    taken as many courses possible in the fields of quantum mechanics,
    electronics, and computers.  This base of knowledge has prepared me well to
    delve deeper into the relationships between them that are essential in the
    development of a quantum computer.  Quantum mechanics is one of my favorite
    subjects, as I have mentioned. My engineering coursework taught me many
    practical skills such as signal processing and an introduction to
    microcontrollers.  In addition to my courses in electrical engineering and
    physics, I have had the opportunity to take additional elective computer
    science courses. This was an opportunity to pursue a personal field of
    interest for me.  This experience both solidified my interest in quantum
    computing and sharpened my independent learning skills. Because I had not
    taken the prerequisite courses for several of these electives, I filled the
    gaps through independent study and leveraging the resources available to
    me. As a result, I have become a much more effective programmer, which has
    in turn made me more effective in both coursework and research.

    My courses have also helped me develop soft skills that are essential to
    success in an academic space. The engineering curriculum requires a two
    semester design project.  My team was rather large, which required careful
    attention to dividing the work equitably while accounting for each
    individual's strengths and weaknesses.  We had no formal team leader, but I
    stepped into a role as a leader on the team, coordinating many of the
    group's efforts.

    My courses have developed resiliency in me. Often, in academia or in any
    field, an approach does not work as expected. Throughout my undergraduate
    education, I have learned to accept these unexpected results, reassess my
    approach, and proceed in a modified direction with confidence.  For
    example, I received a rather poor grade in the first semester of a sequence
    of two electricity and magnetism courses. However, I knew that with
    different study approaches and a little more time, I could learn the
    material better and succeed in the subsequent course. I took the next
    course in the sequence, made the appropriate adjustments in my study
    habits, and I got a much better grade (A-).

    When I first expressed interest in quantum computing, I was told that Yale
    was the best institution for research in that field, and that Professors
    Schoelkopf, Devoret, and Girvin were doing some interesting research. Upon
    reading their webpages, I found this to be quite an understatement. Their
    research in circuit quantum electrodynamics is absolutely fascinating, and
    it is my hope to be able to contribute to this field one day. In addition
    to the inherent wonder of convincing tiny particles to work together and
    behave in a useful way, pursuing a PhD at Yale in the study of circuit
    quantum electrodynamics will enable me to be a leader in the field
    of quantum computing.
}


\end{document}
