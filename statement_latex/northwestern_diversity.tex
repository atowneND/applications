\documentstyle[12pt]{article}
\setlength{\oddsidemargin}{0in}
\setlength{\evensidemargin}{0in}
\setlength{\textwidth}{6.5in}
\setlength{\topmargin}{-.3in}
\setlength{\textheight}{9in}
\pagestyle{empty}

\begin{document}

\begin{center}
{\large Ashley Towne}
\end{center}

\vspace*{.5in}
{Diversity of thought, identity, and experience are integral parts of any
    successful group of people. Appreciating differences in peers, superiors,
    and subordinates can be a challenge, however, especially when common ground
    is difficult to find. The sum of small instances of diversity that make
    each individual unique is where disparate groups of people can find common
    ground.

    Upon first meeting me, one might say that I am different because I am a
    woman in STEM or because I chose to pursue a dual degree with physics and
    electrical engineering. Some people find these characteristics slightly 
    off-putting. These things may distinguish me from a majority of my peers,
    but they do not separate me from people who do not initially appear to be
    similar to me.

    There are a number of life experiences that are outside the norm, each of
    which contributes to a unique perspective that can help me find common
    ground within a diverse group. My dad is in the Army, so as a child I moved
    around a lot. I was born in Germany and moved six times before I was seven
    years old. The deployment of my dad resulted in a temporary single-parent
    household with a unique set of challenges.

    I had the opportunity to be exposed to many different educational
    environments as I attended several different schools, some of which were
    public and others which were private.  My high school had a very diverse
    population, racially, ethnically, religiously, and especially
    socioeconomically. I learned to navigate controversial conversations with
    people who had very different backgrounds and opinions and to connect with
    them through our common interests.

    Transitioning to Notre Dame for college was a bit of a culture shock,
    initially. On the surface, Notre Dame is a relatively homogeneous place.
    However, I have been able to embrace diversity, even when roughly 85\% of
    the students share a common faith. My choice to pursue engineering and
    science is rare at Notre Dame. In order to pursue my goal, I had to fight
    bureaucracy, adjust my academic timeline, and endure personal and academic
    costs. As a woman in engineering and physics, there is some cultural bias
    against me, but I have been very fortunate that the Notre Dame community
    cultivates an environment of respect. I have had the privilege of seeing
    how a historically male-dominated field can successfully transition to an
    environment welcoming to both men and women, from an institutional to an
    individual level.  

    Notre Dame's places great emphasis on pursuing social justice both within the
    university and by reaching out to the communities around us, especially
    those who are sometimes overlooked. My choir at Notre Dame, the Folk Choir,
    has toured overseas in Europe and Australia, as well as domestically in the
    United States. The purpose of touring with the Folk Choir is to build
    community and make connections with people from very different backgrounds.
    Every year that we are able, we also give a concert in a local maximum
    security prison. We sing with the men there and are allowed time to talk to
    them and hear their stories. They are resilient men, each with a unique
    story of determination and hope.

    Music is very important to me. I enjoy singing, playing the piano, learning
    the violin, and composing, but my favorite part is the community that it
    builds. Music can break down walls between people who might not think they
    have much in common. Musicianship and appreciation for new music is common
    to many disparate groups of people. It speaks to everyone in different ways
    while simultaneously bringing them together in a community.

    During a summer REU at Lehigh University, I was once again in a 
    diverse environment. My faith is very important to me, so spending a summer
    with a secular group of physicists promised to be an adventure. I enjoy
    logical, thoughtful discourse, so I was looking forward to hearing my
    peers' opposing views on many subjects. I think a lot of people want
    concrete evidence of God's existence. Some even believe that science and
    faith are at odds with one another. However, it is my belief that they
    complement one another because they answer different questions. One of my
    favorite theological discussions I have ever had was with the group of REU
    students at Lehigh, where nearly everyone in the room disagreed with me.
    The best part was discovering that, in spite of our theological
    differences, we had many similar questions and we could help each other
    learn and grow.
    
    Every individual has some small quirks that make them unique. Finding
    common quirks, whether a hobby or a similar life experience, can unite a
    disparate group of people. My experiences with people from many different
    backgrounds has taught me how to find the small connections and build
    relationships on those. Connecting with others via small diversities also
    leads to a greater appreciation of the bigger diversities.
}


\end{document}
