\documentstyle[12pt]{article}
\setlength{\oddsidemargin}{0in}
\setlength{\evensidemargin}{0in}
\setlength{\textwidth}{6.5in}
\setlength{\topmargin}{-.3in}
\setlength{\textheight}{9in}
\pagestyle{empty}

\begin{document}

\begin{center}
{\Large Personal History} \\[.3in]
{\large Ashley Towne}
\end{center}

\vspace*{.5in}
{My statement of purpose describes my academic motivations for pursuing a
    PhD, but I have personal motivations as well. I relish intellectual and
    personal challenges, especially if they involve situations where no one has
    gone before. My program of study, the dual degree for engineering and
    science, was not a formal program when I decided to study both electrical
    engineering and physics. However, in spite of its rarity - or perhaps
    because of it - I signed up to go against the grain. A PhD is a great
    challenge, and it is the next adventure that I am thrilled to face.

    However, intellectual challenge alone is not enough for me. My engineering
    background and internships in industry have made it clear to me that I find
    great personal fulfillment in working with technology that impacts the way
    the world works. I have done two internships and two summers of research. I
    loved the real world applications in the internships. The knowledge that I
    was making a difference in someone's life was rewarding and fulfilling.
    Quantum computing strikes a delicate balance between interesting physics
    and a useful application that could change the world and how we use and
    think about information. I want to make a difference using the gifts that I
    have been given.

    I know that there will be many challenges in pursuing a PhD, many of which
    I will not be able to anticipate. People keep telling me that, as a woman
    in science and engineering, I will encounter discrimination. I have been
    fortunate since many great people at Notre Dame have paved the way and
    cultivated an environment of respect; I can count the individuals who have
    made sexist remarks on one hand. While I am not naive enough to believe
    that the rest of the world will be as gentle, I have had the privilege to
    experience a community that treats people with respect. I expect and demand
    this level of respect for myself and everyone that I work with. Anything
    less is unacceptable.

    I believe I can make a difference academically in the field of quantum
    computing. My courses and experience have prepared me well enough to
    succeed. However, I can also make a difference personally in the lives of
    those around me. ``If you see something, say something'' is a popular
    security catchphrase, but it also applies to demanding respect for
    everyone. One of the values of the founders of Notre Dame is to make God
    known, loved, and served. The best way to do this is through serving others
    and showing them compassion and respect. Ensuring that the people around me
    receive the respect they deserve is very important to me.  Notre Dame's
    commitment to these values is one of the reasons I decided to go there 
    for my undergraduate studies, and it is a value that I will carry with me
    for the rest of my life, wherever I go. Doing the right thing is not always
    easy, but it is a challenge, and I love to sign up for those.
}
\end{document}
